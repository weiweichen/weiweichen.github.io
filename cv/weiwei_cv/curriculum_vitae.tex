% (c) 2002 Matthew Boedicker <mboedick@mboedick.org> (original author) http://mboedick.org
% (c) 2003-2007 David J. Grant <davidgrant-at-gmail.com> http://www.davidgrant.ca
% (c) 2008-2012 Nathaniel Johnston <nathaniel@njohnston.ca> http://www.njohnston.ca
% (c) 2012	Weiwei Chen <weiwei.chen@uci.edu> http://www.cecs.uci.edu/~weiweic
%
% Depending on your TeX distribution, you may need to download the revnum and longtable packages for this template to work!
%
%This work is licensed under the Creative Commons Attribution-Noncommercial-Share Alike 2.5 License. To view a copy of this license, visit http://creativecommons.org/licenses/by-nc-sa/2.5/ or send a letter to Creative Commons, 543 Howard Street, 5th Floor, San Francisco, California, 94105, USA.

\documentclass[letterpaper,10pt]{article}
\newlength{\outerbordwidth}
\pagestyle{empty}
\raggedbottom
\raggedright
\usepackage[svgnames]{xcolor}
\usepackage{enumerate}
\usepackage{framed, tabularx, array}
\usepackage{longtable}
\usepackage{revnum}
\usepackage[colorlinks=true,urlcolor=blue]{hyperref}
\usepackage{tocloft}
\usepackage{fancyhdr}
\usepackage{verbatim} 


%-----------------------------------------------------------
%Edit these values as you see fit

\setlength{\outerbordwidth}{3pt}  % Width of border outside of title bars
\definecolor{shadecolor}{gray}{0.75}  % Outer background color of title bars (0 = black, 1 = white)
\definecolor{shadecolorB}{gray}{0.93}  % Inner background color of title bars
\renewcommand\familydefault{\sfdefault}
%-----------------------------------------------------------
%Margin setup

\setlength{\evensidemargin}{-0.08in}
\setlength{\headheight}{0in}
\setlength{\headsep}{0in}
\setlength{\oddsidemargin}{-0.08in}
\setlength{\paperheight}{11in}
\setlength{\paperwidth}{8.5in}
\setlength{\tabcolsep}{0in}
\setlength{\textheight}{9in}
\setlength{\textwidth}{6.5in}
\setlength{\topmargin}{-0.3in}
\setlength{\topskip}{0in}
\setlength{\voffset}{0.1in}
\setlength\LTleft{0.3in} % needed to make longtable full-width
\setlength\LTright{0.2in}


%\setlength{\evensidemargin}{-0.25in}
%\setlength{\headheight}{0in}
%\setlength{\headsep}{0in}
%\setlength{\oddsidemargin}{-0.25in}
%\setlength{\paperheight}{11in}
%\setlength{\paperwidth}{8.5in}
%\setlength{\tabcolsep}{0in}
%\setlength{\textheight}{9.5in}
%\setlength{\textwidth}{7in}
%\setlength{\topmargin}{-0.3in}
%\setlength{\topskip}{0in}
%\setlength{\voffset}{0.1in}
%\setlength\LTleft{0.2in} % needed to make longtable full-width
%\setlength\LTright{0.2in}

%-----------------------------------------------------------
%Set header and footer, W.Chen 111212
\fancyhf{} % sets both header and footer to nothing
\renewcommand{\headrulewidth}{0pt}
% your new footer definitions here
\pagestyle{fancy}
\lfoot{Weiwei Chen}
\cfoot{Curriculum Vitae -- \today}
\rfoot{\thepage}

%-----------------------------------------------------------
%Custom commands
\newcommand{\resitem}[1]{\item #1 \vspace{-2pt}}
\newcommand{\resheading}[1]{\vspace{0pt}
  \parbox{\textwidth}{\setlength{\FrameSep}{\fboxsep}
%    \begin{shaded}
%\setlength{\fboxsep}{0pt}\framebox[\textwidth][l]{\setlength{\fboxsep}{4pt}\fcolorbox{shadecolorB}{shadecolorB}{\textbf{\sffamily{\mbox{~}\makebox[6.762in][l]{\Large #1} \vphantom{p\^{E}}}}}}
%\setlength{\fboxsep}{0pt}\framebox[\textwidth][l]{\setlength{\fboxsep}{4pt}\fcolorbox{shadecolorB}{shadecolorB}{\textbf{\sffamily{\mbox{~}\makebox[6.262in][l]{\large #1} \vphantom{p\^{E}}}}}}
\vspace{4mm}
{\textbf{{{\Large #1} \vphantom{p\^{E}}}}}
\vspace{1mm}
%   \end{shaded}
  }\vspace{-5pt}
}

% the next four commands allow for the \ressubheading environment to be 1, 2, 3, or 4 subrows, depending on which command you use. This is admittedly hack-ish, and should probably be replaced by a single more flexible command (with optional arguments) in the future
\newcommand{\ressubheading}[4]{
\begin{tabularx}{6in}[t]{l@{\cftdotfill{\cftsecdotsep}\extracolsep{\fill}}r}
		\textbf{#1} & #2 \\
		\textit{#3} & \textit{#4} \\
\end{tabularx}\vspace{-6pt}}

\newcommand{\ressubheadinga}[2]{
 \begin{tabular*}{6in}[t]{l@{\cftdotfill{\cftsecdotsep}\extracolsep{\fill}}r}
  \textbf{#1} & #2 \\
\end{tabular*}}

\newcommand{\ressubheadingb}[6]{
\begin{tabular*}{6in}[t]{l@{\cftdotfill{\cftsecdotsep}\extracolsep{\fill}}r}
		\textbf{#1} & #2 \\
		\textit{#3} & \textit{#4} \\
		\textit{#5} & \textit{#6} \\
\end{tabular*}}

\newcommand{\ressubheadingc}[8]{
%\begin{tabular*}{6in}[t]{l@{\cftdotfill{\cftsecdotsep}\extracolsep{\fill}}r}
%\begin{tabularx}{6in}[t]{l@{\cftdotfill{\cftsecdotsep}\extracolsep{\fill}}r}
% \begin{tabularx}{6in}[t]{l@{\extracolsep{\fill}}r}
\begin{tabularx}{6in}{X<{\cftdotfill{\cftsecdotsep}\extracolsep{\fill}}@{}r}
  \textbf{#1} & #2 \\
  #3 & #4 \\
  #5 & #6 \\
  #7 & #8 \\
\end{tabularx}}
%\vspace{-6pt}}


%\newcommand{\ressubheadingc}[8]{
%  \begin{tabular*}{6.5in}{l@{\extracolsep{\fill}}r}
%    \textbf{#1} & #2 \\
%    #3 & #4 \\
%    #5 & #6 \\
%    #7 & #8 \\
%  \end{tabular*}\vspace{-6pt}
%}

%\newcommand{\ressubheadingg}[10]{
%\begin{tabular*}{6in}[t]{l@{\cftdotfill{\cftsecdotsep}\extracolsep{\fill}}r}
%		\textbf{#1} & #2 \\
%		\textit{#3} & \textit{#4} \\
%		\textit{#5} & \textit{#6} \\
%		\textit{#7} & \textit{#8} \\
%		\textit{#9} & \textit{#10} \\
%\end{tabular*}}%\vspace{-6pt}}

\newcommand{\ressubheadinge}[8]{
\begin{tabular*}{6in}[t]{l@{\cftdotfill{\cftsecdotsep}\extracolsep{\fill}}l}
		\textbf{#1} & #2 \\
		\textit{#3} & \textit{#4} \\
		\textit{#5} & \textit{#6} \\
		\textit{#7} & \textit{#8} \\
\end{tabular*}\vspace{-6pt}}


\newcommand\foo[9]{%
    \def\tempb{#2}%
    \def\tempc{#3}%
    \def\tempd{#4}%
    \def\tempe{#5}%
    \def\tempf{#6}%
    \def\tempg{#7}%
    \def\temph{#8}%
    \def\tempi{#9}%
    \foocontinued
}
\newcommand\foocontinued[7]{%
    % Do whatever you want with your 9+7 arguments here.
}

\newcommand{\ressubheadingd}[1]{
	\def\argten{#1}%
	\ressubheadingdb
}
\newcommand{\ressubheadingdb}[9]{
\begin{tabular*}{6.5in}[t]{l@{\cftdotfill{\cftsecdotsep}\extracolsep{\fill}}r}
		\textbf{\argten} & #1 \\
		\textit{#2} & \textit{#3} \\
		\textit{#4} & \textit{#5} \\
		\textit{#6} & \textit{#7} \\
		\textit{#8} & \textit{#9} \\
\end{tabular*}\vspace{-6pt}}

\newcommand{\mypubhl}[6]{
{#1}, {``\href{#2}{#3}''}, {#4}{\textit{#5}}, {#6}
}

\newcommand{\mypub}[5]{
{#1}, {{``#2''}}, {#3}{\textit{#4}}, {#5}
}

\newcommand{\mypubs}[4]{
{#1}, {#2}{\textit{#3}}{#4}
}

\newcommand{\ressubheadingf}[2]{
\begin{tabular*}{6.5in}[t]{l@{\cftdotfill{\cftsecdotsep}\extracolsep{\fill}}r}
		\textit{#1} & #2 \\
\end{tabular*}}


%-----------------------------------------------------------


\begin{document}


% contact.tex
%Curriculum Vitae, Weiwei Chen, 11/12/12 first version

% (c) 2002 Matthew Boedicker <mboedick@mboedick.org> (original author) http://mboedick.org
% (c) 2003-2007 David J. Grant <davidgrant-at-gmail.com> http://www.davidgrant.ca
% (c) 2008-2012 Nathaniel Johnston <nathaniel@njohnston.ca> http://www.njohnston.ca
% (c) 2012	Weiwei Chen <weiwei.chen@uci.edu> http://www.cecs.uci.edu/~weiweic
%
% Depending on your TeX distribution, you may need to download the revnum and longtable packages for this template to work!
%
%This work is licensed under the Creative Commons Attribution-Noncommercial-Share Alike 2.5 License. To view a copy of this license, visit http://creativecommons.org/licenses/by-nc-sa/2.5/ or send a letter to Creative Commons, 543 Howard Street, 5th Floor, San Francisco, California, 94105, USA.




{ \begin{tabular*}{4.5in}{l@{\extracolsep{\fill}}l}
\multicolumn{2}{c}{\textbf{\Huge Weiwei Chen}}\\
&\\
Email: \href{mailto:weiwei.chen.uci@gmail.com}{weiwei.chen.uci@gmail.com}& ~~~~Homepage: \hyperref{https://weiweichen.github.io}{}{}{weiweichen.github.io}\\


%~~1645 Villa Street 			& ~~~~~~~~~~~~~~~~~~~~~~~~~~~~Tel: +1(949) 887-6878\\
%~~Mountain View, CA 94041	& ~~~~~~~~~~~~~~~~~~~~~~~~~~~~Email: \href{mailto:weiwei.chen.uci@gmail.com}{weiwei.chen.uci@gmail.com}\\
%~~				& ~~~~~~~~~~~~~~~~~~~~~~~~~~~~Homepage: \hyperref{http://www.cecs.uci.edu/~weiweic}{}{}{www.cecs.uci.edu/$\sim$weiweic}\\


%Department of Electrical Engineering and Computer Science 		& Tel: +1(949) 887-6878\\
%University of California, Irvine								& Email: \href{mailto:weiweic@uci.edu}{weiweic@uci.edu}\\
%Irvine, CA 92697-2625									& \hyperref{http://www.cecs.uci.edu/~weiweic}{}{}{www.cecs.uci.edu/$\sim$weiweic}\\
%~~Qualcomm Research Silicon Valley 			& ~~~~~~~~~~~~~~~~~~~~~~~~~~~~Tel: +1(949) 887-6878\\
%~~SCL.C-180Q, 3165 Kifer Rd			 	& ~~~~~~~~~~~~~~~~~~~~~~~~~~~~Email: \href{mailto:weiweic@uci.edu}{weiweic@uci.edu}, \href{mailto:weiweic@qti.qualcomm.com}%{weiweic@qti.qualcomm.com}\\
%~~Santa Clara, CA 95051					& ~~~~~~~~~~~~~~~~~~~~~~~~~~~~Homepage: \hyperref{http://www.cecs.uci.edu/~weiweic}{}{}{www.cecs.uci.edu/$\sim$weiweic}\\
\end{tabular*}}
\\
\vspace{4mm}

% education.tex
%Curriculum Vitae, Weiwei Chen, 11/12/12 first version

% (c) 2002 Matthew Boedicker <mboedick@mboedick.org> (original author) http://mboedick.org
% (c) 2003-2007 David J. Grant <davidgrant-at-gmail.com> http://www.davidgrant.ca
% (c) 2008-2012 Nathaniel Johnston <nathaniel@njohnston.ca> http://www.njohnston.ca
% (c) 2012	Weiwei Chen <weiwei.chen@uci.edu> http://www.cecs.uci.edu/~weiweic
%
% Depending on your TeX distribution, you may need to download the revnum and longtable packages for this template to work!
%
%This work is licensed under the Creative Commons Attribution-Noncommercial-Share Alike 2.5 License. To view a copy of this license, visit http://creativecommons.org/licenses/by-nc-sa/2.5/ or send a letter to Creative Commons, 543 Howard Street, 5th Floor, San Francisco, California, 94105, USA.


%%%%%%%%%%%%%%%%%%%%%%%%%%%%%%
\resheading{EDUCATION}
%%%%%%%%%%%%%%%%%%%%%%%%%%%%%%

%\begin{itemize}
\begin{enumerate}[ ]
\item 
        \ressubheadingc{Ph.D. Electrical and Computer Engineering}{2013}
	{\textit{Department of Electrical Engineering and Computer Science}}{ }
	{\textit{University of California, Irvine}} { }
	{Dissertation: Out-of-Order Parallel Discrete Event Simulation for ESL Design}{}
	{Committee: Prof. Rainer D\"{o}mer, Prof. Daniel D. Gajski, Prof. Brian Demsky \\} {}
	{\textit{\bf Outstanding Dissertation Award, European Design and Automation Association}}

\item
	\ressubheadingb{M.S. Computer Engineering}{~~~2007} 	              		
	{Thesis: A Symbolic Analog Circuit Simulator}{ }
	%{\textit{School of Microelectronics}}{ }
	{\textit{Shanghai Jiao Tong University, Shanghai, China}}{ }

\item
	\ressubheadingb{B.Eng. Computer Science and Engineering}{2004}
	{Thesis: Design and Implementation of a Software Debugger for Digital Signal Processors}{} 
	%{\textit{Department of Computer Science and Engineering}}{}
	{\textit{Teaching Reform Class (Honor Class with Advanced Admission, National Entrance Exam Waiver)}}{}
	{\textit{Shanghai Jiao Tong University, Shanghai, China}}{}
	
\item
	\ressubheadingc{International Exchange Student}{2003} 			            				
	{Dean's List and Semester Honor}{}
	{\textit{School of Electrical and Computer Engineering}}{}
	{\textit{Purdue University, West Lafayette, Indiana}}{}

%\item
%	\ressubheadinga{High School Graduation}{2000} 
		
%\end{itemize}
\end{enumerate}


% honors.tex
%Curriculum Vitae, Weiwei Chen, 11/12/12 first version

% (c) 2002 Matthew Boedicker <mboedick@mboedick.org> (original author) http://mboedick.org
% (c) 2003-2007 David J. Grant <davidgrant-at-gmail.com> http://www.davidgrant.ca
% (c) 2008-2012 Nathaniel Johnston <nathaniel@njohnston.ca> http://www.njohnston.ca
% (c) 2012	Weiwei Chen <weiwei.chen@uci.edu> http://www.cecs.uci.edu/~weiweic
%
% Depending on your TeX distribution, you may need to download the revnum and longtable packages for this template to work!
%
%This work is licensed under the Creative Commons Attribution-Noncommercial-Share Alike 2.5 License. To view a copy of this license, visit http://creativecommons.org/licenses/by-nc-sa/2.5/ or send a letter to Creative Commons, 543 Howard Street, 5th Floor, San Francisco, California, 94105, USA.


%%%%%%%%%%%%%%%%%%%%%%%%%%%%%%
\resheading{HONORS AND AWARDS}
%%%%%%%%%%%%%%%%%%%%%%%%%%%%%%
\begin{itemize}
\resitem{7 Qualstar Awards, Qualcomm Inc. 2014 - 2016}
\resitem{Outstanding Dissertation Award, European Design and Automation Association (EDAA) 2014}
\resitem{Best Paper Award, Design, Automation and Test Conference in Europe (DATE) 2014}
\resitem{Pedagogical Fellowship, UC Irvine 2012-13}
%\resitem{Young Student Support Award, Design Automation Conference (DAC) 2010}
\resitem{\textit{Henry Samueli} Endowed Fellowship, UC Irvine 2007}
%\resitem{Excellent Teaching Assistant Award, School of Microelectronics, SJTU 2006}
\resitem{National Scholarship for Academic Excellence, China 2006}
\resitem{ \textit{Infineon, Guanghua, Morgan Stanley} Endowed merit-based Scholarship, SJTU 2004-2007}
\resitem{Exceptional Undergraduate Student Awards, SJTU}
\resitem{People's Scholarship for Academic Excellence, SJTU 2000-2004}
\resitem{Fellowship of Pan Wen-Yuan Foundation 2001}
\resitem{Soh Bing (Shu Ping) Scholarship, 9th grade to senior year in college}

\end{itemize}


\newpage

% research.tex
%Curriculum Vitae, Weiwei Chen, 11/12/12 first version

% (c) 2002 Matthew Boedicker <mboedick@mboedick.org> (original author) http://mboedick.org
% (c) 2003-2007 David J. Grant <davidgrant-at-gmail.com> http://www.davidgrant.ca
% (c) 2008-2012 Nathaniel Johnston <nathaniel@njohnston.ca> http://www.njohnston.ca
% (c) 2012	Weiwei Chen <weiwei.chen@uci.edu> http://www.cecs.uci.edu/~weiweic
%
% Depending on your TeX distribution, you may need to download the revnum and longtable packages for this template to work!
%
%This work is licensed under the Creative Commons Attribution-Noncommercial-Share Alike 2.5 License. To view a copy of this license, visit http://creativecommons.org/licenses/by-nc-sa/2.5/ or send a letter to Creative Commons, 543 Howard Street, 5th Floor, San Francisco, California, 94105, USA.

%%%%%%%%%%%%%%%%%%%%%%%%%%%%%%
\resheading{RESEARCH INTERESTS}
%%%%%%%%%%%%%%%%%%%%%%%%%%%%%%
\vspace{1mm}
\begin{itemize}
\item Compiler for ML, HPC, BigData, etc
\item Big Data System Architecture and Acceleration
\item Heterogeneous Parallel Programming and Compilers
\item System-level Modeling, Validation, and Analysis
\end{itemize}
%\begin{center}\begin{longtable}{l@{\extracolsep{\fill}}r}
%		\multicolumn{2}{l}{\textbf{Parallel computing}}\\
%		\multicolumn{2}{l}{\textbf{Design automation for embedded computer systems}}\\
%		\multicolumn{2}{l}{\textbf{Embedded system-level modeling, validation, and analysis}}\\
%		\multicolumn{2}{l}{\textbf{Embedded hardware and software systems}}\\
%\vphantom{E}
%\end{longtable}
%\end{center}\vspace{-36pt}

%%%%%%%%%%%%%%%%%%%%%%%%%%%%%%
\resheading{RESEARCH AND WORKING EXPERIENCE}
%%%%%%%%%%%%%%%%%%%%%%%%%%%%%%
%\textbf{AxStream}
%\ressubheadinga{Software Engineer}{February 2016 -- Present}
%Stealth mode startup for big data acceleration

\textbf{Modular}
\ressubheadinga{Staff Engineer}{May 2023 -- Present}

\vspace{2mm}
Compiler for the \href{https://www.modular.com/mojo}{Mojo} programming language.
\vspace{2mm}

\textbf{SambaNova Systems}
\ressubheadinga{Senior Principal Engineer}{May 2021 -- May 2023}
\ressubheadinga{Principal Engineer}{October 2018 -- April 2021}

\vspace{2mm}
Founding member of the compiler team that builds the compiler stack for the new HW architecture with reconfigurable dataflow units (RDUs): 
MLIR-based graph compiler, kernel compiler, DSLs for RDU, PyTorch framework integration, other SW infrastructure related work.
\vspace{2mm}


\textbf{BigStream}
\ressubheadinga{Founding Software Engineer}{February 2016 -- October 2018}

\vspace{2mm}
Lead software research and development efforts on compilers, native C++ big data acceleration runtime libraries, tool-chains for hardware acceleration (FPGA) technology, and big data system architecture.
\begin{itemize}
\item Native runtime and compiler support for accelerating big data user-defined functions (UDF).
\item A dataflow compiler with Spark SQL frontend, query intermediate representations, three backend for native, FPGA, and RISCV code generation, and a planner for optimized SW/HW accelerator partitioning.
\item A C++ native acceleration library with templatized SQL operations, cluster data source support (HDFS, amazon S3, Microsoft WASB), various input format support (json, avro, parquet, csv, ...), and tensorflow native integration for data pipelines.
\item Spark dataframe and RDD APIs for native and hardware acceleration. 
\item A Clang-LLVM based high-level synthesis compiler for timing scheduling and code generation of FPGA accelerators.
\item On premise Hadoop Cluster setup, and network performance measurement. 
\end{itemize}
\vspace{2mm}

%\begin{itemize}
%\item Spark SQL native software acceleration, including a query compiler, a native C++ acceleration library, integrations of  
%\item Two compilers (in LLVM and Scala) for seamless connection between high-level big data applications with backend acceleration technologies.  
%\item Backend acceleration libraries for big data applications
%\item Network performance tuning and measurement
%\item (Other things and details that I am not allowed to mention before we exit stealth mode)
%\end{itemize}

\textbf{Qualcomm Research Silicon Valley} \\
\ressubheadinga{Senior Engineer}{October 2013 -- Feburary 2016}
%\begin{itemize}

\vspace{2mm}
Parallel programming patterns and runtime for heterogeneous multi-core platforms, 
%  \textit{Qualcomm Symphony System Manager SDK \\
%  (previously know as Qualcomm Multicore Asynchronous Runtime Environment, i.e. MARE,  %\href{http://developer.qualcomm.com/symphony}{http://developer.qualcomm.com/symphony})}
 \textbf{Qualcomm Symphony System Manager SDK} \href{http://developer.qualcomm.com/symphony}{http://developer.qualcomm.com/symphony}

\begin{itemize}
\item Heterogeneous Parallel Pipeline Pattern API and internal scheduling
\item Task and dataflow API infrastructures
\item Parallelize Android native computational photography and enterprise compression applications using task-based parallel programming patterns
\item Power and performance evaluation for native parallel applications
\end{itemize}  
  
%\item 
\vspace{2mm}
Compiler front-end analysis and back-end code generation for coarse-grain auto-parallelization base on Polyhedral Optimizations in LLVM (llvm-polly). 
\vspace{4mm}
%\end{itemize}


\textbf{University of California, Irvine} \\
\ressubheadinga{Graduate Student Researcher, Department of EECS}{September 2007 -- 2013}
\begin{itemize}
\item  {Multi-core parallel simulation for Transaction-Level Models (TLMs)} \\
\item  {Recoding diagnosis for parallel system-level embedded application models}  \\
\item  {A SystemC frond-end using Clang tooling} \\
\item  {System-level modeling and synthesis for parallel embedded standard applications} \\
%\item  {Fast simulation for cyclo-static data flow models } \\
\item  {ConcurrenC: a novel Model of Computation (MoC) for effective system-level abstraction of C-based System-Level Description Languages (SLDLs)} \\
\end{itemize}

\textbf{Microsoft, Redmond, WA}\\
\vspace{-2mm}
\begin{itemize}
\item 
\ressubheadinga{Software Develop Engineer Intern	}{ June 2011 -- September 2011}
\textit{Windows Core Security and Identity Public Key Infrastructure Team} \\
Developed a Windows store application for secure banking with cloud roaming features on the Windows 8 Platform in Javascript
(\hyperref{http://code.msdn.microsoft.com/windowsapps/Metro-style-banking-app-7d963c00}{}{}{Windows 8 banking app with strong authentication sample})
\end{itemize}

\textbf{Shanghai Jiao Tong University}\\
\ressubheadinga{Graduate Research Assistant, School of Microelectronics}{~~~December 2004 -- January 2007}
\begin{itemize}
\item   A symbolic analog circuit simulation using graph reduction approaches
\item   Simulation for heterogeneous multiprocessor systems based on the SimpleScalar toolset
\item   MP3 decoder algorithm optimization for DSP and an in-house operating system on ARM9 platform
\item   Digital circuit design for a reconfigurable cache controller and external memory interface module in VerilogHDL
\end{itemize}

\input{publications}

% professions.tex
%Curriculum Vitae, Weiwei Chen, 11/12/12 first version

% (c) 2002 Matthew Boedicker <mboedick@mboedick.org> (original author) http://mboedick.org
% (c) 2003-2007 David J. Grant <davidgrant-at-gmail.com> http://www.davidgrant.ca
% (c) 2008-2012 Nathaniel Johnston <nathaniel@njohnston.ca> http://www.njohnston.ca
% (c) 2012	Weiwei Chen <weiwei.chen@uci.edu> http://www.cecs.uci.edu/~weiweic
%
% Depending on your TeX distribution, you may need to download the revnum and longtable packages for this template to work!
%
%This work is licensed under the Creative Commons Attribution-Noncommercial-Share Alike 2.5 License. To view a copy of this license, visit http://creativecommons.org/licenses/by-nc-sa/2.5/ or send a letter to Creative Commons, 543 Howard Street, 5th Floor, San Francisco, California, 94105, USA.
%\newpage
%%%%%%%%%%%%%%%%%%%%%%%%%%%%%%
\resheading{PROFESSIONAL ACTIVITIES AND SERVICES}
%%%%%%%%%%%%%%%%%%%%%%%%%%%%%%
%\newpage
\vspace{2mm}

\textbf{\large Conference Reviewer} \\
Expert Reviewer \\
\vspace{-2mm}
\begin{itemize}
\resitem {Design Automation Conference (DAC) 2013}
\end{itemize}

External Reviewer \\
\vspace{-2mm}
\begin{itemize}
\resitem {Design Automation Conference (DAC) 2009, 2010}
\resitem {Design, Automation and Test in Europe Conference (DATE) 2010, 2011, 2013, 2014}
\resitem {ACM/IEEE International Conference on Formal Methods and Models for Co-design (MEMOCODE) 2010}
\resitem {International Conference on Hardware/Software Co-design and System Synthesis (CODES+ISSS) 2010, 2012, 2013}
\resitem {IEEE Symposium on High Performance Computer Architecture (HPCA) 2016}
\end{itemize}

\textbf{\large Book Chapter Reviewer} \\
\vspace{-2mm}
\begin{itemize}
\resitem{Handbook of Hardware/Software Codesign, Springer}
\end{itemize}

\textbf{\large Journal Reviewer} \\
\vspace{-2mm}
\begin{itemize}
\resitem{ACM Transaction on Embedded Computing (TECS)}
\resitem{Springer's Journal of Network and Systems Management}
\resitem{Elsevier's Journal of Simulation Modelling Practice and Theory}
\resitem{Journal of Parallel and Distributed Computing}
\resitem{IEEE Micro}
\resitem{ACM Transactions on Design Automation of Electronic Systems (TODAES)}
\end{itemize}

\textbf{\large Conference Program Committee} \\
\vspace{-2mm}
\begin{itemize}
\resitem {Artifact Evaluation Committee Member \\ International Symposium on Code Generation and Optimization (CGO) 2015, 2016}
\resitem {Area Co-lead \\ Qualcomm Innovation Fellowship - Mobile Application and Apps Enablers}
\end{itemize}

\textbf{\large Professional Association Membership}\\
\vspace{-2mm}
\begin{itemize}
\item ACM, IEEE, IEEE Computer Society 
\end{itemize}

\textbf{\large Conference Presentations}\\
\vspace{-2mm}
\begin{itemize}
\item IESS'09, ASP-DAC'10, ASP-DAC'12, DATE'12, DAC'12, HLDVT'12, DATE'13, DATE'14, PPoPP'15
\end{itemize}

\textbf{\large Invited Talks}\\
\vspace{-2mm}
\begin{enumerate}[\textbf{T\arabic{enumi}.}]
\item
	\mypub
	{Invited Lecture}
	{Discussion for C-based SLDLs: SpecC and SystemC} 
	{}
	{SoC Description and Modeling (EECS 222A)}
	{UC Irvine, December 4, 2009}
	
%\item
%	\mypub
%	{Invited Talk}
%	{Internship in Microsoft}
%	{}
%	{Group Seminar}
%	{Center for Embedded Computer Systems, UC Irvine, October 14, 2011}
	
	
\item Invited Talk, ``Multi-Core Parallel Simulation of System-Level Description Languages'', School of Microelectronics, Shanghai Jiao Tong University, December 26, 2011
	
%\item 
%	\mypub
%	{Contributed to Invited Talk, Rainer D\"{o}mer, Weiwei Chen, Xu Han} 
%	{Advances in Parallel Discrete Event Simulation For Embedded System Design}
%	{}
%	{EECS Colloquium}
%	{UC Irvine, May 9, 2012}
	
%\item 
%	\mypub
%	{Contributed to Invited Talk, Rainer D\"{o}mer, Weiwei Chen, Xu Han}
%	{Advances in Parallel Simulation of System Models}
%	{}
%	{EECS Colloquium} 
%	{UC Irvine, October 17, 2012}
	
\item Invited Talk, ``Out-of-order Parallel Discrete Event Simulation for Electronic System-Level Design'', School of Microelectronics, Shanghai Jiao Tong University, China, December 12, 2012

\item Invited Talk, ``Out-of-order Parallel Simulation for Electronic System-Level Design'', Department of Computer Science, The Carl von Ossietzky University of Oldenburg, Germany, March 14, 2013

\item ``Part II, MARE High-level API"  \href{https://sites.google.com/site/maretutorial2015/}{MARE Tutorial: Power Programming for Mobile Computing}, 20th ACM SIGPLAN Symposium on Principles and Practice of Parallel Programming (PPoPP), San Francisco, February 8, 2015 

\end{enumerate}

%
% teaching.tex
%Curriculum Vitae, Weiwei Chen, 11/12/12 first version

% (c) 2002 Matthew Boedicker <mboedick@mboedick.org> (original author) http://mboedick.org
% (c) 2003-2007 David J. Grant <davidgrant-at-gmail.com> http://www.davidgrant.ca
% (c) 2008-2012 Nathaniel Johnston <nathaniel@njohnston.ca> http://www.njohnston.ca
% (c) 2012	Weiwei Chen <weiwei.chen@uci.edu> http://www.cecs.uci.edu/~weiweic
%
% Depending on your TeX distribution, you may need to download the revnum and longtable packages for this template to work!
%
%This work is licensed under the Creative Commons Attribution-Noncommercial-Share Alike 2.5 License. To view a copy of this license, visit http://creativecommons.org/licenses/by-nc-sa/2.5/ or send a letter to Creative Commons, 543 Howard Street, 5th Floor, San Francisco, California, 94105, USA.

%%%%%%%%%%%%%%%%%%%%%%%%%%%%%%
\resheading{TEACHING EXPERIENCE}
%%%%%%%%%%%%%%%%%%%%%%%%%%%%%%
\textbf{University of California, Irvine} \\
\vspace{-2mm}
\begin{itemize}
\item 
\ressubheadinga{Pedagogical Fellow}{Academic year 2012-13}
\textbf{Teaching, Learning and Technology Center (TLTC)} and \\
\textbf{The Henry Samueli School of Engineering}
\vspace{-2mm}	
\begin{itemize}
\item \textit{Teaching Assistant Professional Development Program (TAPDP 2012, 2013)}\\
Designed and led a day-and-a-half discipline-specific, interactive workshop series to prepare graduate students with their instructional careers in University of California at Irvine. The program included eleven workshops concerning TA responsibilities, learning styles, active learning strategies, problem solving skills, grading, leading discussion sessions, office hours, handling difficult situations, and microteaching
%\vspace{-1mm}	
\item \textit{Teaching Consultation}\\
Conduct teaching consultations with TAs through reflecting on teaching experience, and identifying effective teaching methods and strategies
\end{itemize}

%\newpage
\item 
\ressubheadinga{Teaching Assistant}{} 	
\vspace{-2mm}		
\begin{itemize}
\item \textit{Advanced C Programming (EECS22)}{~~~~~~~~~~~~~~~~~~~~~~~~~~~~~~~~~~~~~~~~~~~~~~~~~~~~~~~~Fall 2011, 2012}
\item \textit{Computational Methods in Electrical and Computer Engineering (EECS10)}\\
{~~~~~~~~~~~~~~~~~~~~~~~~~~~~~~~~~~~~~~~~~~~~~~~~~~~~~~~~~~~~~~~~~~~~~~~~~~~~~Fall 2008, 2009, 2010, Summer 2012}\\
Led discussion and laboratory sections, designed learning activities, prepared and graded programming homework assignments, held office hours, managed online course message board, prepared the course accreditation (ABET) materials
\end{itemize}

\item 
\ressubheadinga{Substitute Lecturer}{}
\vspace{-2mm}	
\begin{itemize}
\item \textit{Advanced System Software (EECS211)}{~~~~~~~~~~~~~~~~~~~~~~~~~~~~~~~~~~~~~~~~~~~~~~~~~~~~~~~~~Winter 2011} \\
Gave two lectures for memory management in computer systems for graduate students
\end{itemize}
\end{itemize}

%% SJTU %%
%\newpage
\textbf{Shanghai Jiao Tong University} \\
\vspace{-2mm}
\begin{itemize}
\item 
\ressubheadinga{Instructor}{June 2006}	
\vspace{-2mm}		
\begin{itemize}
\item \textit{FPGA Training Workshop} \\									     
Gave lectures on embedded system design by using the Embedded Development Kit of Xilinx FPGA, and
designed the laboratory exercises
\end{itemize}

\item 
\ressubheadinga{Teaching Assistant}{Spring 2005 -- Fall 2006}	
\vspace{-2mm}		
\begin{itemize}
\item \textit{Digital Integrated Circuit Design}
\item \textit{Design Automation for Integrated Circuit	}\\
Prepared exam, homework, and laboratory assignments
\item \textit{Embedded System Design}\\
Mentored undergraduate students for embedded system design projects
\end{itemize}
\end{itemize}


%% Future Road College %%
\textbf{Future Road College, Shanghai, China}\\
\vspace{-2mm}
\begin{itemize}
\item 
\ressubheadinga{Visiting Instructor}{July 2006 -- July 2007}
%\vspace{-2mm}	
Taught high school students Calculus, Linear Algebra and Theory of Probability for preparation for SAT-AP test
\end{itemize}

\begin{comment}
%%%%%%%%%%%%%%%%%%%%%%%%%%%%%%
\resheading{FACULTY DEVELOPMENT}
%%%%%%%%%%%%%%%%%%%%%%%%%%%%%%
\textbf{University of California, Irvine} \\
\begin{itemize}
\item 
\ressubheadinga{Advanced Pedagogy Seminar	 (US390A, B, C)}{Fall, Spring 2012, Winter 2013}
Studied the pedagogy in higher education, teaching and learning styles, and out-come driven course design. Constructed a syllabus for a future course. Built and participated in the learning community with other graduate students who consider a career in academia.  

\item 
\ressubheadinga{Workshop: The Seven Principles of Good Practice in Undergraduate Education}{}		
Studied the Seven Principles of Good Practice in Undergraduate Education, a set of educational practices distilled from over fifty years of education research. Explored these principles through discussion and facilitator modeling, and identified opportunities to practice implementing them.

\item 
\ressubheadinga{Workshop: Effective Responses to Student Writing workshop}{}				
Studied the tips and guidelines for maximizing the impact of responses to written assignments, and strategies for effective responses to student work.

\item 
\textbf{Workshop: How Students Undermine Their Own Performance, \\and What Instructors Can Do To Prevent It}\\
Studied the research on how students� conceptions of learning and their own abilities to perform can limit their abilities to succeed in the college classroom. Explored ways to boost student performance by changing their conceptions of themselves as learners, and practiced implementing these methods in a classroom setting.
\end{itemize}
\end{comment}

%
% working.tex
%Curriculum Vitae, Weiwei Chen, 11/12/12 first version

% (c) 2002 Matthew Boedicker <mboedick@mboedick.org> (original author) http://mboedick.org
% (c) 2003-2007 David J. Grant <davidgrant-at-gmail.com> http://www.davidgrant.ca
% (c) 2008-2012 Nathaniel Johnston <nathaniel@njohnston.ca> http://www.njohnston.ca
% (c) 2012	Weiwei Chen <weiwei.chen@uci.edu> http://www.cecs.uci.edu/~weiweic
%
% Depending on your TeX distribution, you may need to download the revnum and longtable packages for this template to work!
%
%This work is licensed under the Creative Commons Attribution-Noncommercial-Share Alike 2.5 License. To view a copy of this license, visit http://creativecommons.org/licenses/by-nc-sa/2.5/ or send a letter to Creative Commons, 543 Howard Street, 5th Floor, San Francisco, California, 94105, USA.

%\newpage
%%%%%%%%%%%%%%%%%%%%%%%%%%%%%%
\resheading{WORKING EXPERIENCE}
%%%%%%%%%%%%%%%%%%%%%%%%%%%%%%

\textbf{Startup}\\
\vspace{-2mm}
\begin{itemize}

\item
\ressubheadinga{Software Engineering}{October 2018 -- Present}
Stealth mode startup for big data acceleration 
\end{itemize}


\item
\ressubheadinga{Software Engineer}{February 2016 -- October 2018}
Stealth mode startup for big data acceleration 
\end{itemize}

\textbf{Qualcomm Research Silicon Valley} \\
\vspace{-2mm}
\begin{itemize}
\item
\ressubheadinga{Senior Engineer}{October 2013 -- February 2016}
Parallel programming and compiler research for heterogeneous multi-core platforms
\end{itemize}

\textbf{Microsoft, Redmond, WA}\\
\vspace{-2mm}
\begin{itemize}
\item 
\ressubheadinga{Software Develop Engineer Intern	}{ June 2011 -- September 2011}
\textit{Windows Core Security and Identity Public Key Infrastructure Team} \\
Developed a Windows store application for secure banking with cloud roaming features on the Windows 8 Platform in Javascript
(\hyperref{http://code.msdn.microsoft.com/windowsapps/Metro-style-banking-app-7d963c00}{}{}{Windows 8 banking app with strong authentication sample})
\end{itemize}

%\newpage
\textbf{IBM China System \& Technology Lab (CSTL), Shanghai, China}\\
\vspace{-2mm}
\begin{itemize}
\item 
\ressubheadinga{R\&D Engineer Intern}{June 2006-- April 2007}
Developed parallel high-performance sorting algorithms on the CELL Broadband Engine platform \\
Research and development for system software for storage devices (C++ and Java) based on 
\hyperref{http://www.opengroup.org/subjectareas/management/openpegasus}{}{}{the OpenPegasus project}
\end{itemize}

%\newpage


%
% miscellaneous.tex
%Curriculum Vitae, Weiwei Chen, 11/12/12 first version

% (c) 2002 Matthew Boedicker <mboedick@mboedick.org> (original author) http://mboedick.org
% (c) 2003-2007 David J. Grant <davidgrant-at-gmail.com> http://www.davidgrant.ca
% (c) 2008-2012 Nathaniel Johnston <nathaniel@njohnston.ca> http://www.njohnston.ca
% (c) 2012	Weiwei Chen <weiwei.chen@uci.edu> http://www.cecs.uci.edu/~weiweic
%
% Depending on your TeX distribution, you may need to download the revnum and longtable packages for this template to work!
%
%This work is licensed under the Creative Commons Attribution-Noncommercial-Share Alike 2.5 License. To view a copy of this license, visit http://creativecommons.org/licenses/by-nc-sa/2.5/ or send a letter to Creative Commons, 543 Howard Street, 5th Floor, San Francisco, California, 94105, USA.

%%%%%%%%%%%%%%%%%%%%%%%%%%%%%%
\resheading{SOFTWARE RELEASES (as one of the contributors)}
%%%%%%%%%%%%%%%%%%%%%%%%%%%%%%
\begin{itemize}
\item \textbf{Qualcomm Symphony System Manager SDK}\\
A developer programming library and API that enables Android native code developers to harness the performance, energy and thermal benefits of multicore SOCs (System on Chip) in smartphones and tablets, 
available at \href{http://developer.qualcomm.com/symphony}{http://developer.qualcomm.com/symphony}

\item \textbf{SpecC compiler version 2.2.2, Developer Release, UC Irvine}\\
Provided the parallel simulation kernel, the out-of-order parallel simulation kernel, the static code analyzer in the compiler, the race condition diagnosis tool, and 
extended the simulator support for the SoC Environment (SCE) toolset

\item \textbf{Recoding tool support, System-on-Chip Description and Modeling course (EECS222A), UC Irvine} \\
Provide the compiler and simulator infrastructure for the Eclipse IDE tool for the recoding projects of this course

\item \textbf{Embedded application models in the example repository for the SoC Environment (SCE) toolset}\\
Designed an H.264 video decoder model (40k+ lines of code), 
a JPEG image encoder (2.5k+ lines of code), 
a video edge detector, and a DES cipher chip model
\end{itemize}

%%%%%%%%%%%%%%%%%%%%%%%%%%%%%%
\resheading{ONLINE INFORMATION}
%%%%%%%%%%%%%%%%%%%%%%%%%%%%%%
\begin{itemize}
\item Office page: \href{http://www.cecs.uci.edu/~weiweic}{http://www.cecs.uci.edu/$\sim$weiweic}
\item Pedagogical Fellowship Program, Teaching Learning and Technology Center (TLTC) UC Irvine: \href{http://www.tltc.uci.edu/pfProgram.html}{http://www.tltc.uci.edu/pfProgram.html}, 
\href{http://www.tltc.uci.edu/teachingAwards2013.html}{http://www.tltc.uci.edu/teachingAwards2013.html}
\item TA Professional Development Program (TAPDP) teaching portfolio: \href{http://www.cecs.uci.edu/~weiweic/teaching.html}{http://www.cecs.uci.edu/$\sim$weiweic/teaching.html}
%\item TAPDP teaching evaluation: \href{http://www.cecs.uci.edu/~weiweic/teaching/TAPDP2012_Feedback.pdf}{http://www.cecs.uci.edu/$\sim$weiweic/teaching/TAPDP2012\_Feedback.pdf}
%\item Teaching evaluation samples: 
%\href{http://www.cecs.uci.edu/~weiweic/teaching/EECS22_F11_MidtermEvaluation.pdf}{EECS22\_F12\_MidtermEvaluation}, 
%\href{http://www.cecs.uci.edu/~weiweic/teaching/CoverSheet_EECS22_F11_MidtermEvaluation.pdf}{EECS22\_F11\_MidtermEvaluation}, 
%\href{http://www.cecs.uci.edu/~weiweic/teaching/CoverSheet_EECS10_F10_FinalEvaluation.pdf}{EECS10\_F10\_FinalEvaluation}, 
%\href{http://www.cecs.uci.edu/~weiweic/teaching/CoverSheet_EECS10_F09_FinalEvaluation.pdf}{EECS10\_F09\_FinalEvaluation}
\end{itemize}



%\newpage
%
% references.tex
%Curriculum Vitae, Weiwei Chen, 11/12/12 first version

% (c) 2002 Matthew Boedicker <mboedick@mboedick.org> (original author) http://mboedick.org
% (c) 2003-2007 David J. Grant <davidgrant-at-gmail.com> http://www.davidgrant.ca
% (c) 2008-2012 Nathaniel Johnston <nathaniel@njohnston.ca> http://www.njohnston.ca
% (c) 2012	Weiwei Chen <weiwei.chen@uci.edu> http://www.cecs.uci.edu/~weiweic
%
% Depending on your TeX distribution, you may need to download the revnum and longtable packages for this template to work!
%
%This work is licensed under the Creative Commons Attribution-Noncommercial-Share Alike 2.5 License. To view a copy of this license, visit http://creativecommons.org/licenses/by-nc-sa/2.5/ or send a letter to Creative Commons, 543 Howard Street, 5th Floor, San Francisco, California, 94105, USA.
%\newpage
%%%%%%%%%%%%%%%%%%%%%%%%%%%%%%
\resheading{REFERENCES}
%%%%%%%%%%%%%%%%%%%%%%%%%%%%%%
\vspace{2mm}

Dr. Rainer D\"{o}mer (Ph.D. Advisor) \\
Professor\\
Electrical Engineering and Computer Science\\
University of California, Irvine\\
\href{mailto:doemer@uci.edu}{doemer@uci.edu}\\

\vspace{5mm}

Dr. Calin Cascaval (Qualcomm Project Director)\\
Director of Compilers and Tools\\
Barefoot Networks \\
\href{mailto:cascaval@acm.org}{cascaval@acm.org}\\

\vspace{5mm}

Dr. Dilma Da Silva \\
Professor, Department Head, \\
Holder of the Ford Motor Company Design Professorship II \\
Department of Computer Science and Engineering \\
Texas A\&M University \\
\href{mailto:dilma@cse.tamu.edu}{dilma@cse.tamu.edu}\\




% references.tex
%Curriculum Vitae, Weiwei Chen, 11/12/12 first version

% (c) 2002 Matthew Boedicker <mboedick@mboedick.org> (original author) http://mboedick.org
% (c) 2003-2007 David J. Grant <davidgrant-at-gmail.com> http://www.davidgrant.ca
% (c) 2008-2012 Nathaniel Johnston <nathaniel@njohnston.ca> http://www.njohnston.ca
% (c) 2012	Weiwei Chen <weiwei.chen@uci.edu> http://www.cecs.uci.edu/~weiweic
%
% Depending on your TeX distribution, you may need to download the revnum and longtable packages for this template to work!
%
%This work is licensed under the Creative Commons Attribution-Noncommercial-Share Alike 2.5 License. To view a copy of this license, visit http://creativecommons.org/licenses/by-nc-sa/2.5/ or send a letter to Creative Commons, 543 Howard Street, 5th Floor, San Francisco, California, 94105, USA.
%\newpage
%%%%%%%%%%%%%%%%%%%%%%%%%%%%%%
\resheading{REFERENCES}
%%%%%%%%%%%%%%%%%%%%%%%%%%%%%%
\begin{comment}
Dr. Rainer D\"{o}mer (Ph.D. Advisor) \\
Professor\\
Electrical Engineering and Computer Science\\
University of California, Irvine\\
+1 (949) 824-9007\\
\href{mailto:doemer@uci.edu}{doemer@uci.edu}\\

\vspace{5mm}

Dr. Daniel Gajski (Dissertation Committee)\\
Professor, Founding Director\\
Center for Embedded Computer Systems\\
University of California, Irvine\\
+1 (949) 824-4155\\
\href{mailto:gajski@uci.edu}{gajski@uci.edu}\\

\vspace{5mm}

Dr. Brian Demsky (Dissertation Committee)\\
Professor\\
Electrical Engineering and Computer Science\\
University of California, Irvine\\
+1 (949) 824-0356\\
\href{mailto:bdemsky@uci.edu}{bdemsky@uci.edu}\\

\vspace{5mm}

Dr. Christopher O'Neal (Pedagogical Fellow Supervisor)\\
Director of Faculty Development \\
David Geffen School of Medicine \\
University of California, Los Angeles\\
+1 (310) 825-8463\\
\href{mailto:coneal@mednet.ucla.edu}{coneal@mednet.ucla.edu}\\

%Associate Director \\
%Teaching, Learning, and Technology Center\\
%University of California, Irvine\\
%+1 (949) 824-6307\\
%\href{mailto:coneal@uci.edu}{coneal@uci.edu}\\
\end{comment}

{\begin{tabular*}{6.5in}{l@{\extracolsep{\fill}}l}
%\multicolumn{2}{c}{\textbf{\Huge Weiwei Chen}}\\

Dr. Rainer D\"{o}mer (Ph.D. Advisor)                                   & Dr. Calin Cascaval \\
Professor                                                                            & Director \\
Electrical Engineering and Computer Science                            &Barefoot Networks, Palo Alto, CA \\
University of California, Irvine                                       &  \href{mailto:cascaval@acm.org}{cascaval@acm.org} \\
+1 (949) 824-9007                                                      & \\
\href{mailto:doemer@uci.edu}{doemer@uci.edu}                           & \\
&\\

Dr. Daniel Gajski (Dissertation Committee)                             & Dr. Brian Demsky (Dissertation Committee) \\
Professor Emeritus, Founding Director                                  & Professor \\
Center for Embedded Computer Systems                                   & Electrical Engineering and Computer Science \\
University of California, Irvine                                       & University of California, Irvine \\
+1 (949) 824-4155                                                      & +1 (949) 824-0356 \\
\href{mailto:gajski@uci.edu}{gajski@uci.edu}                           & \href{mailto:bdemsky@uci.edu}{bdemsky@uci.edu} \\
&\\

Dr. Christopher O'Neal (Pedagogical Fellowship Supervisor)             & Dr. Guoyong Shi (Master's Thesis Advisor)\\
Director of Faculty Development                                        & Professor \\
David Geffen School of Medicine                                        & School of Microelectronics\\
University of California, Los Angeles                                  & Shanghai Jiao Tong University\\
+1 (310) 825-8463                                                      & +86 (21) 34204546 x 1064\\
\href{mailto:coneal@mednet.ucla.edu}{coneal@mednet.ucla.edu}           & \href{mailto:shiguoyong@ic.sjtu.edu.cn}{shiguoyong@ic.sjtu.edu.cn}\\
&\\
\end{tabular*}}



\begin {comment}
Dr. Christopher O'Neal (Pedagogical Fellowship Supervisor)             &Dr. Yongxin Zhu\\
Director of Faculty Development                                        &Associate Professor\\
David Geffen School of Medicine                                        &School of Microelectronics\\
University of California, Los Angeles                                  &Shanghai Jiao Tong University\\
+1 (310) 825-8463                                                      &+86 (21) 34204546 x 1037\\
\href{mailto:coneal@mednet.ucla.edu}{coneal@mednet.ucla.edu}           &\href{mailto:zhuyongxin@sjtu.edu.cn}{zhuyongxin@sjtu.edu.cn}\\
\end{tabular*}}
\end {comment}




\begin{comment}
Dr. Rainer D\"{o}mer (Ph.D. Advisor) 					&	Dr. Christopher M. O'Neal \\
Associate Professor									&	Associate Director \\
Electrical Engineering and Computer Science				&	Teaching, Learning and Technology Center\\
University of California, Irvine							&	University of California, Irvine\\
+1 (949) 824-9007									&	+1 (949) 824-6307\\
\href{mailto:doemer@uci.edu}{doemer@uci.edu}			& 	\href{mailto:coneal@uci.edu}{coneal@uci.edu}\\
&\\

Dr. Daniel D. Gajski 									&	\\
Professor, Director									&	\\
Center for Embedded Computer Systems					&	\\
University of California, Irvine							&	\\
+1 (949) 824-4155									&	\\
\href{mailto:gajski@uci.edu}{gajski@uci.edu}				&	\\
\end{comment}

%
% references.tex
%Curriculum Vitae, Weiwei Chen, 11/12/12 first version

% (c) 2002 Matthew Boedicker <mboedick@mboedick.org> (original author) http://mboedick.org
% (c) 2003-2007 David J. Grant <davidgrant-at-gmail.com> http://www.davidgrant.ca
% (c) 2008-2012 Nathaniel Johnston <nathaniel@njohnston.ca> http://www.njohnston.ca
% (c) 2012	Weiwei Chen <weiwei.chen@uci.edu> http://www.cecs.uci.edu/~weiweic
%
% Depending on your TeX distribution, you may need to download the revnum and longtable packages for this template to work!
%
%This work is licensed under the Creative Commons Attribution-Noncommercial-Share Alike 2.5 License. To view a copy of this license, visit http://creativecommons.org/licenses/by-nc-sa/2.5/ or send a letter to Creative Commons, 543 Howard Street, 5th Floor, San Francisco, California, 94105, USA.
%\newpage
%%%%%%%%%%%%%%%%%%%%%%%%%%%%%%
\resheading{RESEARCH ACCOMPLISHMENTS IN MY FIELD}
%%%%%%%%%%%%%%%%%%%%%%%%%%%%%%

Computers are ubiquitous in our modern society. The rapid growth of application complexities imposes an imperative need on high computational capacities. However, the fundamental physical bottleneck prevents single hardware unit from increasing processing frequency indefinitely for high performance. The shift towards multi-core and many-core systems is inevitable to improve the computing capabilities in the future. 
\bigskip

In order to exploit the multiple computational resources on the hardware platform, software need to be written in the way that many calculations are carried out concurrently. Parallel programming breaks the program into independent pieces so that they can be executed simultaneously on each processing elements. Parallel programing also makes it possible to save energy consumptions by finishing the work on time without increasing the processor frequency. 
\bigskip

Parallel programming is not an easy task since most of the programming languages and existing programs are designed and written in a sequential way. Compiler technologies have been proposed to automatically parallelizing the sequential part of a program to run on multi-core computers. However, existing approaches can only handle some special cases in a program, such as loops. It is still not feasible to parallelize a whole piece of program without any manual work. On the other hand, new programming languages, such as OpenMP, Cilk, CUDA, are proposed so as to write a parallel program from scratch. Parallel programming models, such as  shared memory, message passing, and task parallelism, are also developed to provide better abstractions to the programmer so that they can focus more on designing their algorithms without worrying to much on the underlying parallelizing details on the hardware platform. 
\bigskip

Parallel computing is pervasive in almost all the computational applications, including scientific computations, networking systems, computer vision, machine learning, as well as mobile consumer applications. It is a collaboration work among advanced hardware design, operating system support, compiler technology and parallel algorithms for domain specific applications. Parallel computing is one of the most fundamental and needed technologies for tomorrow's computational world.  

%%%%%%%%%%%%%%%%%%%%%%%%%%%%%%
%\resheading{RESEARCH ACCOMPLISHMENTS IN MY FIELD}
\resheading{VALUES OF BENEFITS OF MY PROJECTS}
%%%%%%%%%%%%%%%%%%%%%%%%%%%%%%
My research work falls into the realm of parallel computing for embedded and mobile computer systems. 

\bigskip
My Ph.D. dissertation research has been focusing on parallel simulation approaches for system-level description languages (SLDLs) in which most embedded system models are described. 
Embedded computing systems have a profound impact on our everyday life with a wide application domain. More than 90\% of the computer systems in the world are now embedded. Embedded systems are usually designed for a specific purpose with very strict design requirements. To design an embedded system, engineers start from capturing the critical features of the design in an abstract model, and then refine it step by step to final the implementation. Typically, these models are written in SLDLs and validated through simulation. 

\bigskip
My Ph.D. work extents the existing sequential discrete event simulation kernel for SLDLs to support parallel simulation on multicore simulation hosts. In addition to the parallel extension, I also proposed the out-of-order parallel simulation approach to parallelize the simulation more aggressively without loss of accuracy. My work makes it possible for SLDL simulators to exploit the multicore computational capability that are commonly available nowadays in a most efficient way. It has been published in more than 10 peer-reviewed papers and acknowledged by the design automation community with an outstanding dissertation award and one best paper award. A research project has also been funded by Intel to port this technology into industrial usage. 

\bigskip
Now I am working with the Qualcomm Research Silicon Valley on the project of Multicore Asynchronous Runtime Environment (MARE). Our work aims at providing a general parallel programming model for programmers to easily design their algorithms and fully exploit the potential computational resources on multicore platforms. We are building the task-based parallel infrastructure and designing parallel programming patterns to capture the high-level essential abstractions of common algorithms, including linear algebra computation, image processing and other parallel patterns. With our runtime library, engineers can focus on designing their parallel algorithms without being distracted by the complexity of the traditional multithreading programming paradigm and be able to optimize their code on hardware platforms with heterogenous computing units including CPUs, graphics processing unit (GPU), digital signal processor (DSP) and other hardware accelerators. 

\bigskip
Parallel programming is the cornerstone for the future computation world. It brings the great potential of high execution performance and low energy consumption. The contributions of my research work can help the engineers to build and validate their designs in an efficient way which lead to lower production prices and shorter time-to-market. The engineers can therefore be freed from dealing with complex low-level implementation details but focus on designing their algorithms to bring more applications with sophisticated functionalities. In other words, my work is promising to benefit the national interests with advancing computational technologies, reduced energy usage, and improving economy with more efficient consumer and industrial electronics. 




%\newpage
%
% references.tex
%Curriculum Vitae, Weiwei Chen, 11/12/12 first version

% (c) 2002 Matthew Boedicker <mboedick@mboedick.org> (original author) http://mboedick.org
% (c) 2003-2007 David J. Grant <davidgrant-at-gmail.com> http://www.davidgrant.ca
% (c) 2008-2012 Nathaniel Johnston <nathaniel@njohnston.ca> http://www.njohnston.ca
% (c) 2012	Weiwei Chen <weiwei.chen@uci.edu> http://www.cecs.uci.edu/~weiweic
%
% Depending on your TeX distribution, you may need to download the revnum and longtable packages for this template to work!
%
%This work is licensed under the Creative Commons Attribution-Noncommercial-Share Alike 2.5 License. To view a copy of this license, visit http://creativecommons.org/licenses/by-nc-sa/2.5/ or send a letter to Creative Commons, 543 Howard Street, 5th Floor, San Francisco, California, 94105, USA.
%\newpage
%%%%%%%%%%%%%%%%%%%%%%%%%%%%%%
\resheading{POTENTIAL REFERENCES}
%%%%%%%%%%%%%%%%%%%%%%%%%%%%%%
Dr. Rainer D\"{o}mer (Ph.D. Advisor) \\
Associate Professor\\
Electrical Engineering and Computer Science\\
University of California, Irvine\\
%+1 (949) 824-9007\\
%\href{mailto:doemer@uci.edu}{doemer@uci.edu}\\
\vspace{5mm}

Dr. Daniel Gajski \\
Professor, Founding Director\\
Center for Embedded Computer Systems\\
University of California, Irvine\\
%+1 (949) 824-4155\\
%\href{mailto:gajski@uci.edu}{gajski@uci.edu}\\
\vspace{5mm}

Dr. Fadi Kurdahi\\
Professor, Director\\
Center for Embedded Computer Systems\\
University of California, Irvine\\
\vspace{5mm}

Dr. Calin Cascaval\\
Senior Director, Engineering \\
Qualcomm Research Silicon Valley \\
\vspace{5mm}

Prof. Dr. Wolfgang Rosenstiel\\
Dean, Faculty of Science \\
University of T\"{u}bingen, Germany \\
\vspace{5mm}

Prof. Dr.-Ing. Wolfgang Nebel \\
Professor \\
Computer Science (Informatik) Department \\
The Carl von Ossietzky University of Oldenburg, Germany\\
\vspace{5mm}

Prof. Sridevan Parameswaran \\
Professor,  Program Director  for Computer Engineering \\
School of Computer Science and Engineering \\
University of New South Wales, Australia \\
\vspace{5mm}

Dr. Ajit Dingankar \\
Principal Engineer at Intel \\
\vspace{5mm}

Dr. Desmond Kirkpatrick \\
Principal Engineer at Intel \\
\vspace{5mm}

Prof. Petru Eles \\
Professor of Embedded Computer Systems, Deputy Chairman\\
Department of Computer and Information Science (IDA) \\
Link\"{o}ping University,  Sweden\\
\vspace{5mm}

Prof. Soonhoi Ha \\
Professor \\
Computer Engineering Department \\
Seoul National University, South Korea \\
\vspace{5mm}

Prof. J\"{o}rg Henkel \\
Professor, Chair for Embedded Systems \\
Karlsruhe Institute of Technology, Germany \\
\vspace{5mm}

Prof. Lothar Thiele \\
Professor  \\
Computer Engineering and Networks Laboratory  \\
Swiss Federal Institute of Technology Zurich, Switzerland \\
\vspace{5mm}

Prof. Dr.-Ing. Dr. h.c. Gerhard Fettweis  \\
Vodafone Chair Mobile Communications Systems \\
Dresden University of Technology \\
\vspace{5mm}

Prof. Giovanni De Micheli  \\
Professor and Director  \\
The Institute of Electrical Engineering and The Integrated Systems Centre \\
Swiss Federal Institute of Technology in Lausanne (EPFL), Switzerland \\
\vspace{5mm}

Prof. Dr.-Ing. J\"{u}rgen Teich \\
Professor, Chair for Hardware-Software-Co-Design \\
Department of Computer Science \\
University of Erlangen-Nuremberg, Germany \\
\vspace{5mm}

Prof. Dr. Peter Marwedel \\
Professor   \\
Head of Design Automation for Embedded Systems Group  \\
Technical University of Dortmund  \\
\vspace{5mm}


Prof. Dr. Oliver Bringmann \\
Professor, Chair  \\
Embedded Systems at the Department of Computer Science  \\
The University of T\"{u}bingen, Germany \\
Director  \\
FZI Research Center for Information Technologies in Karlsruhe, Germany \\
\vspace{5mm}



%%%%%%%%%%%%%%%%%%%%%%%%%%%%%%
%\resheading{Technical Skills}
%%%%%%%%%%%%%%%%%%%%%%%%%%%%%%
%\begin{itemize}
%\item
%	Markup Languages
%	\begin{itemize}
%		\resitem{CSS, \LaTeX, (X)HTML}
%	\end{itemize}
%
%\item
%	Programming Languages
%	\begin{itemize}
%		\resitem{ASP, C, Java, Javascript, PHP, Python, SQL, Visual Basic}
%	\end{itemize}
%
%\item
%	Specialized Software
%	\begin{itemize}
%		\resitem{Maple, MATLAB, S-Plus}
%	\end{itemize}
%\end{itemize}
\end{document}